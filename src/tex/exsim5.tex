% Pacotes e configurações padrão do estilo ``article''\
% -------------------------------------
\documentclass[a4paper,11pt]{article}
% Layout
% ------------------------------------------------------------------------------
\input{relat_layout.tex}

%\usepackage{circuitikz}
\usepackage[makestderr]{pythontex}

\title{Laboratório 5} % Define o título do Relatório
\author{Rafael Lima}

% Definições Auxiliares ( Macros próprias )
% ------------------------------------------------------------------------------
%\input{relat_aux.tex} % Arquivo com minhas macros
\newcommand{\npy}[1]{\sympy{round(#1,4)}}
% ----------------------------------~>ø<~---------------------------------------
\begin{document}
% Capa e Índice ----------------------------------------------------------------
\input{relat_capa.tex} % Capa para UnB
% Conteúdo ---------------------------------------------------------------------

\section{Projeto de Controlador Deadbeat}

% Código fonte colocado a parte para facilitar validação dentro do ipython
\begin{sympycode}
# Get Source Code
sys.path.insert(1, '../../')
from src.python.exsim5 import *
\end{sympycode}


\subsection{Part 1}

\begin{equation}\label{eq:ex5-g1}
    G_1(z) = \sympy{sG1}
\end{equation}

Para um controlador \textit{deadbeat} é buscado o erro zero para uma determinada entrada em tempo finito. Para tal podemos definir a função de transferência de malha fechada $M_1(z)$ desejada:

\begin{equation}\label{eq:ex5-gmf1}
    M(z) = \frac{A(z)}{z^n}
\end{equation}

Isolando $G_c(z)$ temos

\begin{equation}\label{eq:ex5-gc1}
    G_{c}(z) = \frac{1}{G(z)}\frac{M(z)}{1-M(z)}
\end{equation}

Para que o sistema seja realizavel é necessário que que $n \ge \#polos - \#zeros = 4  - 2 = 2$. Substituindo em \ref{eq:ex5-gmf1} temos

$$
M_1(z) = \sympy{roundExpr(sM1)}
$$

Substituindo $M(z) = M_1(z)$ na expressão \ref{eq:ex5-gc1}:

$$G_{c1}(z) = \sympy{roundExpr(sD1)}$$

Expandindo os termos:

$$G_{c1}(z) = \sympy{roundExpr(simplifyFraction(sD1))}$$

Com isto foi obtido a seguinte resposta ao degrau unitário:

\begin{figure}[H]
    \centering
    \includegraphics[width=0.6\linewidth]{img/exsim5-g1-deadbeat-sim.png}
    \caption{Resposta do Sistema para o controlador Deadbeat para sistema \ref{eq:ex5-g1}}
\end{figure}

\subsection{Part 2}

Temos que $n = \#polos - \#zeros + \#zeros = 3  - 2 + 2 = 1$

\begin{equation}\label{eq:ex5-g2}
    G_2(z) = \sympy{sG2}
\end{equation}

A partir do qual foi obtido a seguinte resposta em malha fechada para a rampa unitária:

\begin{figure}[H]
    \centering
    \includegraphics[width=0.6\linewidth]{img/exsim5-g2-deadbeat-sim.png}
    \caption{Resposta do Sistema para o controlador Deadbeat para sistema \ref{eq:ex5-g1}}
\end{figure}

\subsection{Part 1}

\begin{figure}[H]
    \centering
    \includegraphics[width=0.9\linewidth]{img/exsim5model.png}
    \caption{Diagrama de Blocos do Sistema no Simulink - Planta exsim5}
\end{figure}


\subsubsection{}

Para o fins de comparação foi adotado a seguinte diagrama para o controle projetados para o exercício de simulação 2:

\begin{figure}[H]
    \centering
    \includegraphics[width=1\linewidth]{img/exsim2model.png}
    \caption{Diagrama de Blocos do Sistema no Simulink - Planta exsim2}
\end{figure}

Com base nos valores definidos em script, foram obtidos os seguintes resultados para a simulação:

\begin{figure}[H]
    \centering
    \begin{subfigure}[m]{0.49\linewidth}
        \centering
        \includegraphics[width=1\linewidth]{img/exsim5-deadbeat-sim.png}
        \caption{Controlador Deadbeat - exsim5}
    \end{subfigure}
    \hfill
    \begin{subfigure}[m]{0.49\linewidth}
        \centering
        \includegraphics[width=1\linewidth]{img/exsim5-exsim2-sim.png}
        \caption{Controlador Discretizado - exsim2}
    \end{subfigure}
\end{figure}

Comparando ambas simulações percebemos que a metodologia adotada para o projeto do controlador na lanta exsim2 acaba levando mais tempo para estabilizar e têm um atraso em relação ao sistema contínuo. Enquanto para o controle da planta exsim5 é percebido uma resposta mais rápida e também uma ação de controle bme mais agressiva para um mesmo tempo de amostragem contando com uma varição bastante brusca do sinal passado para a planta. Algo que é característico de controladores deadbeat.

\section{Conclusão}


% ------------------------------------------------------------------------------
\newpage
% Referências
\addcontentsline{toc}{section}{Referências} % Adiciona linha no indice
\bibliographystyle{abbrv} % Define Estilo e gera bibliografia
\bibliography{references} % Adiciona Arquivo com Referências

% Acrescentadas no arquivo references.bib
% para usa-las no texto basta usar \citep{}
% para citar sem usar no texto basta usar \nocite{}
\nocite{sympy}
\nocite{pythontex}
\nocite{matlabcontrol}
\nocite{matlabsymbolic}
\nocite{ogata2010modern}

% ------------------------------------------------------------------------------
\newpage
\section*{Anexos}
\addcontentsline{toc}{section}{Anexos} % Adiciona linha no indice
\subsection*{Python}

Para os cálculos e demonstrações foi utilizado o pacote \textit{Python}\TeX\ \cite{pythontex} para o \LaTeX\ em conjunto da bibliteca \textit{sympy}\cite{sympy}. Segue o script completo em python:

\inputminted[xleftmargin=15pt,linenos,frame=single,framesep=5pt,breaklines=true]{python}{../python/exsim5.py}

\newpage
\subsection*{Matlab}

\subsubsection*{Parte 1}
Para o desenho dos gráficos e simulações foi utilizado o \textit{Matlab} em conjunto das toolbox \textit{Control System}\cite{matlabcontrol} e \textit{Symbolic Math}\cite{matlabsymbolic}. Segue o código referente usado

\inputminted[xleftmargin=15pt,linenos,frame=single,framesep=5pt,breaklines=true]{matlab}{../matlab/exsim5/exsim5.m}

\subsubsection*{Parte 3 - Controlador Dead Beat}
Para avaliação da resposta do controlador Dead Beat foi utilizado uma versão modificada dos script \textit{exsim5script} em \textit{Matlab} fornecido pelo professor:
\inputminted[xleftmargin=15pt,linenos,frame=single,framesep=5pt,breaklines=true]{matlab}{../matlab/exsim5/exsim5script.m}

\subsubsection*{Parte 3 - Controlador Discretizado}
Para comparativo com os resultados da simulação 2, foi utilizado uma versão modificada dos script \textit{exsim2script} em \textit{Matlab} fornecido pelo professor:
\inputminted[xleftmargin=15pt,linenos,frame=single,framesep=5pt,breaklines=true]{matlab}{../matlab/exsim5/exsim2script.m}



% ------------------------------------------------------------------------------
\end{document}
