% Pacotes e configurações padrão do estilo ``article''\
% -------------------------------------
\documentclass[a4paper,11pt]{article}
% Layout
% ------------------------------------------------------------------------------
\input{relat_layout.tex}

%\usepackage{circuitikz}
\usepackage[makestderr]{pythontex}

\title{Laboratório 5} % Define o título do Relatório
\author{Rafael Lima}

% Definições Auxiliares ( Macros próprias )
% ------------------------------------------------------------------------------
%\input{relat_aux.tex} % Arquivo com minhas macros
\newcommand{\npy}[1]{\sympy{round(#1,4)}}
% ----------------------------------~>ø<~---------------------------------------
\begin{document}
% Capa e Índice ----------------------------------------------------------------
\input{relat_capa.tex} % Capa para UnB
% Conteúdo ---------------------------------------------------------------------

\section{Projeto de Controlador Deadbeat}

% Código fonte colocado a parte para facilitar validação dentro do ipython
\begin{sympycode}
# Get Source Code
sys.path.insert(1, '../../')
from src.python.exsim5 import *
\end{sympycode}


\subsection{Part 1}

Dado o sistema definido pela seguinte função de tranferência:

\begin{equation}\label{eq:ex5-g1}
    G(z) = \frac{b(z)}{a(z)} = \sympy{sG1}
\end{equation}

Com a seguinte estrutura em malha fechada, em que $G_c(z)$ é um controlador discreto.

\begin{equation}
    M(z) = \frac{G_c(z)G(z)}{1 - G_c(z)G(z)}
\end{equation}

Para um controlador \textit{deadbeat} é buscado o erro zero para uma determinada entrada em tempo finito. Para tal podemos definir a função de transferência de malha fechada $M_1(z)$ desejada:

\begin{equation}\label{eq:ex5-gmf1}
    M(z) = \frac{A(z)}{z^n}
\end{equation}

Aplicando o teorema do valor final temos que o erro em regime permanente para a resposta ao degrau é dado por

$$E_{ss} = lim_{z \rightarrow 1}(z - 1)\frac{A(z)}{z^n}\frac{z}{(z-1)}$$

A partir do qual temos que $E_{ss} = 1 \Leftrightarrow A(z) = 1$. Assim podemos escolher $A(z) = 1$ para uma resposta em tempo mínimo, de modo que teremos uma resposta deadbeat no instantes amostrados. No entanto esta escolha acaba tendo como efeito a presentação de oscilação entre o instantes amostrados. Para evitar este problema foi adotado $A(z) = \frac{b(z)}{b(1}$ que representa um função de transferência normalizada em $z=1$ com os mesmos zeros que $G_1(z)$. Em outras palavras será adotado um contralador que tenha como efeito apenas alterar a resposta em regime permanente e a localização dos polos em malha fechada mas que conserve os mesmos zeros que a planta sem controlador.Para isto, isolando $G_c(z)$ na equação \ref{eq:ex5-gmf1} obtemos

$$
    G_{c}(z) = \frac{1}{G(z)}\frac{M(z)}{1-M(z)}
$$

Como $G(z) = \frac{b(z)}{a(z)}$, logo para a resposta em malha fechada escolhida teremos

\begin{equation}\label{eq:ex5-gc1}
    G_{c}(z) = \frac{a(z)}{b(z)}\frac{\frac{b(z)}{b(1)}}{z^n-\frac{b(z)}{b(1)}}
\end{equation}


Para que o sistema seja realizável é necessário que que $n \ge \#polos - \#zeros = 4  - 2 = 2$. Substituindo em \ref{eq:ex5-gmf1} temos

$$
M_1(z) = \sympy{roundExpr(sM1)}
$$

Substituindo $M(z) = M_1(z)$ na expressão \ref{eq:ex5-gc1}:

$$G_{c1}(z) = \sympy{roundExpr(sD1)}$$

Expandindo os termos:

$$G_{c1}(z) = \sympy{roundExpr(simplifyFraction(sD1))}$$

Com isto foi obtido a seguinte resposta ao degrau unitário:

\begin{figure}[H]
    \centering
    \includegraphics[width=0.6\linewidth]{img/exsim5-g1-deadbeat-sim.png}
    \caption{Resposta do Sistema para o controlador Deadbeat para sistema \ref{eq:ex5-g1}}
\end{figure}

\subsection{Part 2}


\begin{equation}\label{eq:ex5-g2}
    G_2(z) = \sympy{sG2}
\end{equation}

Procedendo de forma similar, foi adotado a seguinte resposta em malha fechada

$$
    M_2(z) = \frac{(n+1)z -n}{z^{n+1}}
$$

Temos que $n = \#polos - \#zeros + \#zeros = 3  - 2 + 2 = 1$, logo:

\begin{equation}\label{eq:ex5-g2}
    M_2(z) = \sympy{sM2}
\end{equation}

$$M_2(z) = \sympy{roundExpr(sM2)}$$

Substituindo $M(z) = M_2(z)$ na expressão \ref{eq:ex5-gc2}:

$$G_{c2}(z) = \sympy{roundExpr(sD2)}$$

Expandindo os termos:

$$G_{c2}(z) = \sympy{roundExpr(simplifyFraction(sD2))}$$

A partir do qual foi obtido a seguinte resposta em malha fechada para a rampa unitária:

\begin{figure}[H]
    \centering
    \includegraphics[width=0.6\linewidth]{img/exsim5-g2-deadbeat-sim.png}
    \caption{Resposta do Sistema para o controlador Deadbeat para sistema \ref{eq:ex5-g1}}
\end{figure}

\subsection{Part 3}

\subsubsection{Sistema em Malha Fechada}

Dado $G(s)$

\begin{equation}
    G(s) = \frac{Y(s)}{U(s)} \sympy{G22}
\end{equation}

\begin{equation}
    U(s) = \sympy{G21}U_C(s) - \sympy{H22}Y(s)
\end{equation}

Aplicando o sinal de controle e fechando a malha obtemos:

$$G_{mf}(s) = \frac{Y(s)}{U_c(s)} = \sympy{G22mf}$$

Substituindo $a = \sympy{2*w0}$ e $b = \sympy{w0/2}$ temos

$$
G_{mf}(s) = \sympy{G22mf.subs([(a,2*w0),(b,w0/2)])}
$$
$$
G_{mf}(s) = \sympy{G22mf.subs([(a,2*w0),(b,w0/2),(kc,2*J*w0*w0/kp)])}
$$

Simplificando

\begin{equation}
    G_{mf}(s) = \sympy{simplifyFraction(G22mf.subs([(a,2*w0),(b,w0/2),(kc,2*J*w0*w0/kp)]),s)}
\end{equation}

\subsubsection{}

A discretização pode ser aproximada aplicando-se a transformada Z em conjunto de um segurador de ordem zero em série com o sistema. Desta forma temos $G_o(z) = Z(G_{zoh}(s)G(s)$ onde $G_{zoh}(s) = \frac{1-e^{-Ts}}{s}$. Aplicando as propriedades da transformada Z:

$$
\begin{array}{lcl}
    Z(\frac{1-e^{-Ts}}{s}G(z)) &=& Z(\frac{1}{s}G(z)) - Z(\frac{e^{-Ts}}{s}G(z))\\
    &=& Z(\frac{1}{s}G(z)) - (z^{-1})Z(\frac{1}{s}G(z))\\
    &=& (1 - z^{-1})Z(\frac{1}{s}G(z))\\
    &=& (1 - z^{-1})Z(\sympy{(1/s)*G22})\\
\end{array}
$$

Logo

\begin{equation}\label{eq:ex5-partialfrac}
 G(z) =   (1 - z^{-1})Z(\sympy{G22/s})
\end{equation}

Pela tabela temos que a transformada z de \ref{eq:ex5-partialfrac} é

$$G(z) = \sympy{Gz22} = \sympy{sGz22}$$
$$G(z) = \frac{T^2 k_{P}}{2J}\frac{z(1+z^{-1})}{z^2(1-z^{-1})^2}$$

\begin{equation}
    G(z) =\left(\frac{T^2 k_{P}}{2J}\right)\frac{z^{-1}(1+z^{-1})}{(1-z^{-1})^2}
\end{equation}

Como $G(z) = \frac{Y(z)}{U(z)}$

$$
\frac{Y(z)}{U(z)} =\left(\frac{T^2 k_{P}}{2J}\right)\frac{(z^{-1} + z^{-2})}{(1 - 2 z^{-1} + z^{-2})}
$$

$$
 Y(z)\left(1 - 2 z^{-1} + z^{-2}\right)
 = \left(\frac{T^2 k_{P}}{2J}\right)U(z)\left(z^{-1} + z^{-2}\right)
$$

De modo que temos a seguinte equação de diferenças:

\begin{equation}
 Y[k] +  2Y[k -1] +  Y[k - 2]
 = \left(\frac{T^2 k_{P}}{2J}\right)\left(U[k -1] +  U[k - 2]\right)
\end{equation}

\subsubsection{}

\begin{figure}[H]
    \centering
    \includegraphics[width=0.9\linewidth]{img/exsim5model.png}
    \caption{Diagrama de Blocos do Sistema no Simulink - Planta exsim5}
\end{figure}

\subsubsection{Comparação Controle Deadbeat vs Controle Discratizado}

Para o fins de comparação foi adotado a seguinte diagrama para o controle projetados para o exercício de simulação 2:

\begin{figure}[H]
    \centering
    \includegraphics[width=1\linewidth]{img/exsim2model.png}
    \caption{Diagrama de Blocos do Sistema no Simulink - Planta exsim2}
\end{figure}

Com base nos valores definidos em script, foram obtidos os seguintes resultados para a simulação:

\begin{figure}[H]
    \centering
    \begin{subfigure}[m]{0.49\linewidth}
        \centering
        \includegraphics[width=1\linewidth]{img/exsim5-deadbeat-sim.png}
        \caption{Controlador Deadbeat - exsim5}
    \end{subfigure}
    \hfill
    \begin{subfigure}[m]{0.49\linewidth}
        \centering
        \includegraphics[width=1\linewidth]{img/exsim5-exsim2-sim.png}
        \caption{Controlador Discretizado - exsim2}
    \end{subfigure}
\end{figure}

Comparando ambas simulações percebemos que a metodologia adotada para o projeto do controlador na lanta exsim2 acaba levando mais tempo para estabilizar e têm um atraso em relação ao sistema contínuo. Enquanto para o controle da planta exsim5 é percebido uma resposta mais rápida e também uma ação de controle bme mais agressiva para um mesmo tempo de amostragem contando com uma varição bastante brusca do sinal passado para a planta. Algo que é característico de controladores deadbeat.

\section{Conclusão}

O controlador deadbeat constitui uma opção interessante por garantir de em um tempo finito o erro $0$ em regime permanente para uma determinada entrada de referência. Constindo por tanto uma opção bastante interessante para projetos de posicionamento como servo mecanismos ao custo que acaba trazendo uma resposta bastante agressiva para o controlador, podendo provocar um desgate do atuador caso seja especificado tempos de amostragem muito curtos.

% ------------------------------------------------------------------------------
\newpage
% Referências
\addcontentsline{toc}{section}{Referências} % Adiciona linha no indice
\bibliographystyle{abbrv} % Define Estilo e gera bibliografia
\bibliography{references} % Adiciona Arquivo com Referências

% Acrescentadas no arquivo references.bib
% para usa-las no texto basta usar \citep{}
% para citar sem usar no texto basta usar \nocite{}
\nocite{sympy}
\nocite{pythontex}
\nocite{matlabcontrol}
\nocite{matlabsymbolic}
\nocite{ogata2010modern}

% ------------------------------------------------------------------------------
\newpage
\section*{Anexos}
\addcontentsline{toc}{section}{Anexos} % Adiciona linha no indice
\subsection*{Python}

Para os cálculos e demonstrações foi utilizado o pacote \textit{Python}\TeX\ \cite{pythontex} para o \LaTeX\ em conjunto da bibliteca \textit{sympy}\cite{sympy}. Segue o script completo em python:

\inputminted[xleftmargin=15pt,linenos,frame=single,framesep=5pt,breaklines=true]{python}{../python/exsim5.py}

\newpage
\subsection*{Matlab}

\subsubsection*{Parte 1}
Para o desenho dos gráficos e simulações foi utilizado o \textit{Matlab} em conjunto das toolbox \textit{Control System}\cite{matlabcontrol} e \textit{Symbolic Math}\cite{matlabsymbolic}. Segue o código referente usado

\inputminted[xleftmargin=15pt,linenos,frame=single,framesep=5pt,breaklines=true]{matlab}{../matlab/exsim5/exsim5.m}

\newpage
\subsubsection*{Parte 3 - Controlador Dead Beat}
Para avaliação da resposta do controlador Dead Beat foi utilizado uma versão modificada dos script \textit{exsim5script} em \textit{Matlab} fornecido pelo professor:
\inputminted[xleftmargin=15pt,linenos,frame=single,framesep=5pt,breaklines=true]{matlab}{../matlab/exsim5/exsim5script.m}

\newpage
\subsubsection*{Parte 3 - Controlador Discretizado}
Para comparativo com os resultados da simulação 2, foi utilizado uma versão modificada dos script \textit{exsim2script} em \textit{Matlab} fornecido pelo professor:
\inputminted[xleftmargin=15pt,linenos,frame=single,framesep=5pt,breaklines=true]{matlab}{../matlab/exsim5/exsim2script.m}



% ------------------------------------------------------------------------------
\end{document}
