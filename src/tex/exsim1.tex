% Pacotes e configurações padrão do estilo ``article''\
% -------------------------------------
\documentclass[a4paper,11pt]{article}
% Layout
% ------------------------------------------------------------------------------
\input{relat_layout.tex}

\usepackage{circuitikz}
\usepackage[makestderr]{pythontex}

\title{Laboratório 1} % Define o título do Relatório
\author{Rafael Lima}

% Definições Auxiliares ( Macros próprias )
% ------------------------------------------------------------------------------
%\input{relat_aux.tex} % Arquivo com minhas macros
% ----------------------------------~>ø<~---------------------------------------
\begin{document}
% Capa e Índice ----------------------------------------------------------------
\input{relat_capa.tex} % Capa para UnB
% Conteúdo ---------------------------------------------------------------------

\section{}

\begin{sympycode}
sys.path.insert(1, '../../')
from src.python.exsim1 import *
\end{sympycode}

Considerando um sistema de controle a tempo discreto com realimentação unitária e período de amostragem $T = 1s$ cuja função de transferência a malha aberta é dado pela equação \ref{eq:openloop-sys} :


\begin{equation}\label{eq:openloop-sys}
    G(z) = \sympy{nGo}
\end{equation}

Denominando $\alpha_1 = 0.3679$, $\alpha_0 = 0.2642$, $\beta_1 = 1$ temos

\begin{equation}\label{eq:openloop-sys-symbolic}
    G(z) = \sympy{sGo}
\end{equation}

Temos que a função de transferência em malha fechada do sistema é

$$G_{mf}(z) = \frac{G(z)}{1-G(z)}$$

\begin{equation}\label{eq:closedloop-sys-symbolic}
    G_{mf}(z) = \sympy{sGc}
\end{equation}

Substituindo $\alpha_1 = 0.3679$, $\alpha_0 = 0.2642$, $\beta_1 = 1$ temos

\begin{equation}\label{eq:closedloop-sys}
    G_{mf}(z) = \sympy{nGc}
\end{equation}

A partir de \ref{eq:closedloop-sys-symbolic} temos que o polinômio característico de malha fechada é

\begin{equation}\label{eq:poly-sys-symbolic}
    G_{mf}(z) = \sympy{poly}
\end{equation}

\section{Estabilidade pelo Critério de Juri}

\subsection{Definição Critério de Jury}

Para um sistema de ordem 2 as codições para estabilidade pelo critério de juri são

\begin{equation}\label{eq:jury-c1}
    \phi(1) > 0
\end{equation}

\begin{equation}\label{eq:jury-c2}
    (-1)^n\phi(-1) > 0\ ,\ n=2
\end{equation}

\begin{equation}\label{eq:jury-c3}
    |A_0| < |A_n|
\end{equation}

\subsection{Avaliando primeira condição}

$$\phi(z) = \sympy{poly}$$
$$\phi(1) = \sympy{eqJ1}$$

\begin{equation}\label{eq:k-jury-c1}
    K < \sympy{K1}
\end{equation}

\subsection{Avaliando primeira condição}

$$\phi(z) = \sympy{poly}$$

$$(-1)^2\phi(-1) = \phi(-1)$$
$$\phi(-1) = \sympy{eqJ2}$$

\begin{equation}\label{eq:k-jury-c2}
    K < \sympy{K2}
\end{equation}

\subsection{Avaliação Estabilidade}

\begin{equation}\label{eq:k-jury-c3}
    K < \sympy{K3}
\end{equation}

\section{Estabilidade pelo Critério de Routh modificado}


\section{Conclusão}

% ------------------------------------------------------------------------------

\bibliographystyle{abbrv}
\bibliography{references}
% Referências
% Acrescentadas no arquivo references.bib
% para usa-las no texto batsa usar \citep{}


% ------------------------------------------------------------------------------
\end{document}
