% Pacotes e configurações padrão do estilo ``article''\
% -------------------------------------
\documentclass[a4paper,11pt]{article}
% Layout
% ------------------------------------------------------------------------------
\input{relat_layout.tex}

\usepackage{circuitikz}
\usepackage[makestderr]{pythontex}

\title{Laboratório 1} % Define o título do Relatório
\author{Rafael Lima}

% Definições Auxiliares ( Macros próprias )
% ------------------------------------------------------------------------------
%\input{relat_aux.tex} % Arquivo com minhas macros
\newcommand{\npy}[1]{\sympy{round(#1,4)}}
% ----------------------------------~>ø<~---------------------------------------
\begin{document}
% Capa e Índice ----------------------------------------------------------------
\input{relat_capa.tex} % Capa para UnB
% Conteúdo ---------------------------------------------------------------------

\section{Análise Sistema Discreto}

% Código fonte colocado a parte para facilitar validação dentro do ipython
\begin{sympycode}
# Get Source Code
sys.path.insert(1, '../../')
from src.python.exsim1 import *
\end{sympycode}

Considerando um sistema de controle a tempo discreto com realimentação unitária e período de amostragem $T = 1s$ cuja função de transferência a malha aberta é dado pela equação \ref{eq:openloop-sys} :

\begin{equation}\label{eq:openloop-sys}
    G(z) = \sympy{nGo}
\end{equation}

Denominando $\alpha_1 = 0.3679$, $\alpha_0 = 0.2642$, $\beta_1 = 1$ temos

\begin{equation}\label{eq:openloop-sys-symbolic}
    G(z) = \sympy{sGo}
\end{equation}

Temos que a função de transferência em malha fechada do sistema é

$$G_{mf}(z) = \frac{G(z)}{G(z) + 1}$$

\begin{equation}\label{eq:closedloop-sys-symbolic}
    G_{mf}(z) = \sympy{sGc}
\end{equation}

A partir de \ref{eq:closedloop-sys-symbolic} temos que o polinômio característico de malha fechada é

\begin{equation}\label{eq:poly-sys-symbolic}
    \phi(z) = \sympy{poly}
\end{equation}

Substituindo $\sympy{a1} = \sympy{na1}$, $\sympy{a0} = \sympy{na0}$, $\sympy{b1} = \sympy{nb1}$ temos, respectivamente para o a função de transferência de malha fechada e o polinônio caractarístico as equações \ref{eq:closedloop-sys} e \ref{eq:poly-sys}.

\begin{equation}\label{eq:closedloop-sys}
    G_{mf}(z) = \sympy{nGc}
\end{equation}

\begin{equation}\label{eq:poly-sys}
    \phi(z) = \sympy{npoly}
\end{equation}

\section{Estabilidade pelo Critério de Juri}

\subsection{Definição Critério de Jury}

Para um sistema de ordem 2 as condições para estabilidade pelo critério de juri, é necessário avaliar 2+1=3 condições para verificar a estabilidade, sendo estas as seguintes:

\begin{equation}\label{eq:jury-c1}
    \phi(1) > 0
\end{equation}

\begin{equation}\label{eq:jury-c2}
    (-1)^n\phi(-1) > 0\ ,\ n=2
\end{equation}

\begin{equation}\label{eq:jury-c3}
    |A_0| < |A_n|
\end{equation}

\subsection{Avaliando primeira condição}

Calculando $\phi(1)$:

$$\phi(z) = \sympy{poly}$$
$$\phi(1) = \sympy{eqJ1}$$
$$\phi(1) = \sympy{neqJ1}$$

Se $\phi(1) > 0$ então

$$\sympy{neqJ1} > 0$$

Isolando $K$:

\begin{equation}\label{eq:k-jury-c1}
    K > \sympy{K1}
\end{equation}

Substituindo $\sympy{a1} = \sympy{na1}$, $\sympy{a0} = \sympy{na0}$, $\sympy{b1} = \sympy{nb1}$ temos

$$K > \sympy{sK1}$$

Logo a primeira condição é

\begin{equation}\label{eq:k-jury-c1-numeric}
    K > \npy{nK1.simplify()}
\end{equation}

\subsection{Avaliando segunda condição}

Dado a equação característica $\phi(z) = \sympy{poly}$ temos

$$(-1)^2\phi(-1) = \phi(-1) = \sympy{eqJ2}$$

Logo

$$\sympy{eqJ2.factor(K)} > 0$$

Isolando $K$

\begin{equation}\label{eq:k-jury-c2}
    K > \sympy{K2}
\end{equation}

Substituindo $\sympy{a1} = \sympy{na1}$, $\sympy{a0} = \sympy{na0}$, $\sympy{b1} = \sympy{nb1}$ temos

$$K < \sympy{sK2}$$

\begin{equation}\label{eq:k-jury-c2-numeric}
    K < \npy{nK2}
\end{equation}

\subsection{Avaliação terceira condição}

$$\phi(z) = \sympy{poly.expand().factor(z)}$$

$$
\left\{\begin{array}{l}
A_0 = \sympy{apoly[0]} = \sympy{napoly[0]}\\
A_1 = \sympy{apoly[1]} = \sympy{napoly[1]}\\
A_2 = \sympy{apoly[2]} = \sympy{napoly[2]}\\
\end{array}
\right.
$$

$$|A_0| < |A_2|$$
$$|\sympy{napoly[0]}| < |\sympy{napoly[2]}|$$
$$0 < |\sympy{napoly[2]}| - |\sympy{napoly[0]}|$$

Isolando $K$

\begin{equation}\label{eq:k-jury-c3-numeric}
    K < \npy{nK3}
\end{equation}

\subsection{Estabilidade}

Comparando o resultado das inequações \ref{eq:k-jury-c1-numeric}, \ref{eq:k-jury-c2-numeric} e \ref{eq:k-jury-c3-numeric} temos que o intervalo de restrição para K é

\begin{equation}\label{eq:k-jury-result}
    K < \sympy{round(min(nK2,nK3),4)}
\end{equation}

\section{Estabilidade pelo Critério de Routh modificado}

O critério de Roth originalmente foi concebido para a avaliação de estabilidade de sistemas contínuos. No entanto podemos adaptar o sistema em tempo discreto a adotando a seguinte substituição de variável

\begin{equation}\label{eq:bilinear-transf-w}
    w = \frac{z+1}{z-1}
\end{equation}

Isolando $z$ temos:

\begin{equation}\label{eq:bilinear-transf-z}
    z = \frac{w+1}{w-1}
\end{equation}

O que representar uma propriedade interessante da transformada bilinear. Substituindo \ref{eq:bilinear-transf-z} em

$$G(w) = \sympy{sGw}$$
$$G(w) = \sympy{sGw.simplify()}$$

Substituindo $\sympy{a1} = \sympy{na1}$, $\sympy{a0} = \sympy{na0}$, $\sympy{b1} = \sympy{nb1}$ temos
$$G(w) = \sympy{nGww}$$

A partir do qual temos a seguinte equação característica no $plano-w$:

\begin{equation}
    \phi_w(w) = \sympy{npolyw}
\end{equation}

% TODO Gerar tabela de Roth
% Em particular para um polinomio de grau dois basta olhar somente ultimo termo

A partir da tabela de Roth obtemos a seguinte expressão $0 < \sympy{sK4}$. Isolando $K$:

\begin{equation}\label{eq:k-roth-numeric}
    K < \npy{nK4}
\end{equation}

Comparando a expressão da inequação \ref{eq:k-roth-numeric} e \ref{eq:k-jury-result}, observa-se que foi obtido um valor similar para ambos métodos.

\section{Resposta do Sistema}

\subsection{Região de Oscilação Constante}

Com base no resultado do critério de estabilidade de Juri, se definimos $K = \sympy{K3}= \npy{nK3}$. Substituindo $K$ na função de transferência temos:

\begin{equation}\label{eq:closedloop-sys-oscilation}
    G_{2}(z) = \sympy{nGc.subs(K,nK4).simplify()}
\end{equation}

Que possui os polos 
$$p_0 = \sympy{opoles[0]}$$
$$p_1 = \sympy{opoles[1]}$$.

Desta forma podemos obter a frequência de oscilação do sistema como $f = \frac{w}{2\pi}$ em que $w = ||p_0|| = ||p_1||$.

$$f = \frac{||p_0||}{2\pi} = \frac{\npy{sympy.arg(opoles[1])}\ rad/s}{2\pi} = \npy{freq} Hz$$

Na figura \ref{fig:ex1-plot-oscilation} temos a resposta ao degrau junto de uma senoide com frequência $f = \npy{freq} Hz$. Note a resposta segue a aproximadamente a mesma forma da senoide em diversos pontos.

\begin{figure}[H]
    \label{fig:ex1-plot-oscilation}
    \centering
    \includegraphics[width=0.9\linewidth]{img/exsim1-plot-oscilation.png}
    \caption{Resposta ao Degrau Unitário do Sistema com $K = \npy{nK3}$}
\end{figure}

\subsection{Sistema Estável}

De forma similar, pelo critério de Juri temos que, para todo valor de $0< K < \npy{nK3}$ o sistema é estável. Em particular para $K = \npy{0.5*nK3}$ temos a seguinte resposta

\begin{figure}[H]
    \centering
    \includegraphics[width=0.9\linewidth]{img/exsim1-plot-stable.png}
    \caption{Resposta ao Degrau Unitário do Sistema com $K = \npy{0.5*nK3}$}
\end{figure}

\subsection{Sistema Instável}

Por fim, para todo valor de $K > \npy{nK3}$ o sistema é instável. Em particular para $K = \npy{2*nK3}$ temos

\begin{figure}[H]
    \centering
    \includegraphics[width=0.9\linewidth]{img/exsim1-plot-instable.png}
    \caption{Resposta ao Degrau Unitário do Sistema com $K = \npy{2*nK3}$}
\end{figure}


\section{Conclusão}



% ------------------------------------------------------------------------------

\nocite{sympy}
\bibliographystyle{abbrv}
\bibliography{references}
% Referências
% Acrescentadas no arquivo references.bib
% para usa-las no texto batsa usar \citep{}

% ------------------------------------------------------------------------------
\section{Anexos}
\subsection{Python}

Para o avaliação da estabilidade usando o critério de Jury e o critério de Roth modificado foi utilizado o seguinte código em python:

\inputminted[xleftmargin=15pt,linenos,frame=single,framesep=5pt]{python}{../python/exsim1.py}

\subsection{Octave}

Para o desenho dos gráficos e simulações foi utlizado o \textit{octave} em conjunto do pacote \textit{control}. Segue o código referente para as simulações

\inputminted[xleftmargin=15pt,linenos,frame=single,framesep=5pt]{matlab}{../matlab/exsim1.m}


% ------------------------------------------------------------------------------
\end{document}
