% Pacotes e configurações padrão do estilo ``article''\
% -------------------------------------
\documentclass[a4paper,11pt]{article}
% Layout
% ------------------------------------------------------------------------------
\input{relat_layout.tex}

%\usepackage{circuitikz}
\usepackage[makestderr]{pythontex}

\title{Laboratório 4} % Define o título do Relatório
\author{Rafael Lima}

% Definições Auxiliares ( Macros próprias )
% ------------------------------------------------------------------------------
%\input{relat_aux.tex} % Arquivo com minhas macros
\newcommand{\npy}[1]{\sympy{round(#1,4)}}
% ----------------------------------~>ø<~---------------------------------------
\begin{document}
% Capa e Índice ----------------------------------------------------------------
\input{relat_capa.tex} % Capa para UnB
% Conteúdo ---------------------------------------------------------------------

\section{Projeto de Controlador na Frequência}

% Código fonte colocado a parte para facilitar validação dentro do ipython
\begin{sympycode}
# Get Source Code
sys.path.insert(1, '../../')
from src.python.exsim4 import *
\end{sympycode}

\subsection{Transformada z de G(s) com segurador de ordem zero em série}

A discretização pode ser aproximada aplicando-se a transformada z em conjunto de um segurador de ordem zero em série com o sistema. desta forma temos $G_{ho}(z) = Z(G_{zoh}(s)G(s)$ onde $G_{zoh}(s) = \frac{1-e^{-ts}}{s}$. aplicando as propriedades da transformada z:

$$
\begin{array}{lcl}
    Z(\frac{1-e^{-ts}}{s}G(z)) &=& Z(\frac{1}{s}G(z)) - Z(\frac{e^{-ts}}{s}G(z))\\
    &=& z(\frac{1}{s}G(z)) - (z^{-1})Z(\frac{1}{s}G(z))\\
    &=& (1 - z^{-1})Z(\frac{1}{s}G(z))\\
    &=& (1 - z^{-1})Z(\sympy{(1/s)*sG})\\
\end{array}
$$

para facilitar o cálculo, fatorando o termo $\sympy{(1/s)*sG}$ pelo método de frações parciais:

\begin{equation}\label{eq:ex4-partialfrac}
\frac{G(s)}{s} = \frac{1}{s+1}-\frac{1}{s}+\frac{1}{s^2}
\end{equation}

pela tabela, temos que a transformada z de \ref{eq:ex4-partialfrac} é

$$G(z) = \sympy{sGz}$$
$$G(z) = \sympy{sGz.combsimp()}$$

logo, substituindo $t = \sympy{nT}$ e simplificando temos:

\begin{equation}
    G(z) = \sympy{roundExpr(sGz.combsimp().subs(T,nT))}
\end{equation}

\subsection{Transformada w}

Substituindo $z = \sympy{zw}$

$$
G(w) = G(z)|_{z=\sympy{zw}} = \sympy{sGw}
$$

substituindo $t = \sympy{nT}$ e simplificando temos

\begin{equation}
    G(w) = \sympy{roundExpr(sGw.combsimp().subs(T,nT))}
\end{equation}

\subsection{Calculo ganho planta}

$$E_v = lim_{w \rightarrow 0}\ w G(w)$$
$$E_v = lim_{w \rightarrow 0}\  \sympy{w*(sGw.combsimp())}$$
$$E_v = lim_{w \rightarrow 0}\  \sympy{exprLimEv}$$
$$E_v = \sympy{Ev}$$

Como $K_v = \frac{1}{E_v} = \sympy{1/Ev}$  e $K_v = 2$ então $K = \sympy{nK}$

\subsection{Diagrama de Bode}

\begin{figure}[H]
    \centering
    \includegraphics[width=0.9\linewidth]{img/exsim4-bodeplot-gw.png}
    \caption{Diagrama de Bode para $G(w)$}
\end{figure}

\section{Conclusão}


% ------------------------------------------------------------------------------
\newpage
% Referências
\addcontentsline{toc}{section}{Referências} % Adiciona linha no indice
\bibliographystyle{abbrv} % Define Estilo e gera bibliografia
\bibliography{references} % Adiciona Arquivo com Referências

% Acrescentadas no arquivo references.bib
% para usa-las no texto basta usar \citep{}
% para citar sem usar no texto basta usar \nocite{}
\nocite{sympy}
\nocite{pythontex}
\nocite{matlabcontrol}
\nocite{matlabsymbolic}

% ------------------------------------------------------------------------------
\newpage
\section*{Anexos}
\addcontentsline{toc}{section}{Anexos} % Adiciona linha no indice
%\subsection*{Python}

%Para os cálculos e demonstrações foi utilizado o pacote \textit{Python}\TeX\ \cite{pythontex} para o \LaTeX\ em conjunto da bibliteca \textit{sympy}\cite{sympy}. Segue o script completo em python:

%\inputminted[xleftmargin=15pt,linenos,frame=single,framesep=5pt,breaklines=true]{python}{../python/exsim4.py}

\newpage
\subsection*{Matlab}

%\subsubsection*{Parte 1}
Para o desenho dos gráficos e simulações foi utilizado o \textit{Matlab} em conjunto das toolbox \textit{Control System}\cite{matlabcontrol} e \textit{Symbolic Math}\cite{matlabsymbolic}. Segue o código referente usado

\inputminted[xleftmargin=15pt,linenos,frame=single,framesep=5pt,breaklines=true]{matlab}{../matlab/exsim4/exsim4.m}

%\subsubsection*{Parte 2}
%Na segunda parte foi utilizado uma versão modificada do script em \textit{Matlab} fornecido pelo professor:
%\inputminted[xleftmargin=15pt,linenos,frame=single,framesep=5pt]{matlab}{../matlab/exsim2/exsim2script.m}



% ------------------------------------------------------------------------------
\end{document}
