% Pacotes e configurações padrão do estilo ``article''\
% -------------------------------------
\documentclass[a4paper,11pt]{article}
% Layout
% --------------------------------------------------------------------------------
\input{relat_layout.tex}
\usepackage{circuitikz}
\usepackage[makestderr]{pythontex}
%\restartpythontexsession{\thesection}

\newcommand{\tituloRelatorio}{Exercícios Presença}
\title{\tituloRelatorio}
\hypersetup{pdftitle={\tituloRelatorio}}% title
\author{Rafael Lima}

% Definições Auxiliares
% --------------------------------------------------------------------------------
%\input{relat_aux.tex} % Arquivo com minhas macros
%\renewcommand{\thesection}{Questão \arabic{section}}
%\renewcommand{\thesubsection}{(\alph{subsection})}
\newcommand{\npy}[1]{\sympy{round(n#1,4)}}

% ----------------------------------~>ø<~---------------------------------------
\begin{document}
% Capa e Índice ---------------------------------------------------------------
%\input{relat_capa.tex} % Capa para UnB
% Conteúdo -------------------------------------------------------------------

\section{Aula 2020-08-18}
\subsection{Exercício}
Considere o exemplo dado em aula de controle do cabeçote de um HD:

$$
\left\{
\begin{array}{l}
    u(t) = K\left(\frac{b}{a}U_c(t) - y(t) + x(t)\right)\\
    \frac{dx}{dt} = -ax + (a-b)y
\end{array}
\right.
$$

Se fosse utilizada a aproximação $\frac{dx}{dt} = \frac{x(t)- x(t-T)}{T}$. Obtenha a nova lei de controle e o pseudo-código correspondente que deve ser implementado em um processador digital de sinais.

\subsection{Resposta}

A partir da equação $\frac{dx}{dt} = -ax + (a-b)y$, aproximando a derivada, temos:

$$\frac{x(t) - x(t-T)}{T} = (a-b) y(t) - a x(t)$$

Isolando $y(t)$:
$$x(t) - x(t-T) = T (a-b) y(t) - T a x(t)$$
$$x(t)( 1 + T a) - x(t-T) = T (a-b) y(t)$$
$$y(t) =\frac{x(t)( 1 + T a) - x(t-T)}{T (a-b)}$$

Discretizando:
\begin{equation}
Y[k] = X[k]\frac{( 1 + T a)}{{T (a-b)}} - X[k-1]\frac{1}{{T (a-b)}}
\end{equation}
\section{Aula 2020-08-20}
 \subsection{Exercício}
 Um sina $y(t)$ é gerado pela convolução de um sinal de banda limitada $x_1(t)$ com outro sinal de banda limitada $x_2(t)$, ou seja $y(t)=x_1(t)*x_2(t)$ sendo $X_1(j\omega) = 0$ para $|\omega| \geq 1000\pi$ e $X_2(j\omega) = 0$ para $|\omega| \geq 2000\pi$. A amostragem com trem de impulsos é realizada sobre $y(t)$ para obter
 
$$y^*(t) = \sum^{+\infty}_{n=-\infty} y(nT)\delta(t-nT)$$

Especifique o intervalo de valores para o período de amostragem $T$ que garante $y(t)$ é recuperável a partir de $y*(t)$:

$$y(t)=x_1(t)*x_2(t)$$

\subsection{Resposta}

Aplicando a transformada, temos

$$y^*(t)=x_1^*(t)x_2^*(t)$$

Pelo teorema de Nyquist a frequência de amostragem têm que ser maior que o dobro da maior frequência do sistema. Como $\omega_1 \leq \omega_2$ então a frequência $w_n > 2w_2$. Logo o período de amostragem máximo será:

$$
\begin{array}{lcl}
     T_0 &=& \frac{2\pi}{\omega_n} = \frac{2\pi}{2\omega_2}  \\
     T_0 &=& \frac{2\pi}{4000\pi rad/s} = 0.5ms
\end{array}
$$

E desta forma o período de amostragem deve estar dentro do intervalo $0< T < 0.5 ms$.

% ---------------------------------------------------------------------------------------

%\bibliographystyle{abbrv}
%\bibliography{references}
% Referências
% Acrescentadas no arquivo references.bib
% para usa-las no texto batsa usar \citep{}


% ---------------------------------------------------------------------------------------
\end{document}
