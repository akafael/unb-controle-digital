% Pacotes e configurações padrão do estilo ``article''\
% -------------------------------------
\documentclass[a4paper,11pt]{article}
% Layout
% --------------------------------------------------------------------------------
\input{relat_layout.tex}
\usepackage{circuitikz}
\usepackage[makestderr]{pythontex}
%\restartpythontexsession{\thesection}

\newcommand{\tituloRelatorio}{Lista 1}
\title{\tituloRelatorio}
\hypersetup{pdftitle={\tituloRelatorio}}% title
\author{Rafael Lima}

% Definições Auxiliares
% --------------------------------------------------------------------------------
%\input{relat_aux.tex} % Arquivo com minhas macros
\renewcommand{\thesection}{Questão \arabic{section}}
\renewcommand{\thesubsection}{(\alph{subsection})}
\newcommand{\npy}[1]{\sympy{round(n#1,4)}}

% ----------------------------------~>ø<~---------------------------------------
\begin{document}
% Capa e Índice ---------------------------------------------------------------
\input{relat_capa.tex} % Capa para UnB
% Conteúdo -------------------------------------------------------------------

\section{}

\begin{sympycode}

# Symbolic
t = Symbol('t') # Tempo
L = Symbol('L') # Espessura
D = Symbol('D') # Diametro
Qf = Symbol('Q_f') # Energia Gasta Pelo fogao
Q = Symbol('Q') # Energia Recebida pela panela
h = Symbol('h') # coeficiente conveccao
Tamb = Symbol('T_a')
Tf = Symbol('T_f')


a = .85

# Calculo area fundo da panela
A = pi*D*D/4

dQ = h*A*(-Tamb+Tf)
\end{sympycode}

\begin{equation}
 \sympy{Derivative(Q,t)} = \sympy{dQ}
\end{equation}

% ---------------------------------------------------------------------------------------

\bibliographystyle{abbrv}
\bibliography{references}
% Referências
% Acrescentadas no arquivo references.bib
% para usa-las no texto batsa usar \citep{}


% ---------------------------------------------------------------------------------------
\end{document}
