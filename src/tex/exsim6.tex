% Pacotes e configurações padrão do estilo ``article''\
% -------------------------------------
\documentclass[a4paper,11pt]{article}
% Layout
% ------------------------------------------------------------------------------
\input{relat_layout.tex}

%\usepackage{circuitikz}
\usepackage[makestderr]{pythontex}

\title{Laboratório 6} % Define o título do Relatório
\author{Rafael Lima}

% Definições Auxiliares ( Macros próprias )
% ------------------------------------------------------------------------------
%\input{relat_aux.tex} % Arquivo com minhas macros
\newcommand{\npy}[1]{\sympy{round(#1,4)}}
% ----------------------------------~>ø<~---------------------------------------
\begin{document}
% Capa e Índice ----------------------------------------------------------------
\input{relat_capa.tex} % Capa para UnB
% Conteúdo ---------------------------------------------------------------------

\section{Projeto de Controlador no Espaço de Estados}

% Código fonte colocado a parte para facilitar validação dentro do ipython
\begin{sympycode}
# Get Source Code
sys.path.insert(1, '../../')
from src.python.exsim6 import *
\end{sympycode}

Dado o sistema descrito por

$$ x[k+1] = \sympy{G} x[x] + \sympy{H} u[k] $$
$$ y[k] = \sympy{C} x[x] $$

Onde $G=\sympy{G}$, $H=\sympy{H}$ e $C=\sympy{C}$.

\subsection{Calculo ganho feedback}

Supondo $F = \sympy{F}$

$$A_{ma} = \sympy{z*eye(2)} - \sympy{G} + \sympy{H}\sympy{F}$$
$$A_{ma} = \sympy{z*eye(2)} - \sympy{G -H*F}$$
$$A_{ma} = \sympy{z*eye(2) - (G -H*F)}$$

Logo temos que o polinômio característico para o sistema com realimentação é dado por 

$$\phi(z) = det\sympy{Ama}$$
$$\phi(z) = \sympy{det(Ama)}$$

Temos que o polinômio caracteristico desejado é dado por

$$\phi_d(z) = \sympy{polyDesired}$$
$$\phi_d(z) = \sympy{simplify(polyDesired)}$$

Se $\phi(z) = \phi_d(z)$ logo

$$
\left\{
    \begin{array}{l}
        \sympy{polyPlant.coeff(z,0)} = \sympy{polyDesired.coeff(z,0)}\\
        \sympy{polyPlant.coeff(z,1)} = \sympy{polyDesired.coeff(z,1)}\\
        \sympy{polyPlant.coeff(z,2)} = \sympy{polyDesired.coeff(z,2)}\\
    \end{array}
\right.
$$

Resolvendo o sistema temos $f_1 = \sympy{nf1} $ e $f_2 = \sympy{nf2}$

\subsection{Calculo ganho observador polos iguais}

Supondo $L = \sympy{L}$

$$A_{ma} = \sympy{z*eye(2)} - \sympy{G} + \sympy{L}\sympy{C}$$
$$A_{ma} = \sympy{z*eye(2)} - \sympy{G -L*C}$$
$$A_{ma} = \sympy{z*eye(2) - (G -L*C)}$$

Logo temos que o polinômio característico para o sistema com realimentação é dado por 

$$\phi(z) = det\sympy{Amo}$$
$$\phi(z) = \sympy{det(Amo)}$$
$$\phi(z) = \sympy{collect(det(Amo),z)}$$

Temos que o polinômio caracteristico desejado é dado por

$$\phi_d(z) = \sympy{polyObsDesired}$$
$$\phi_d(z) = \sympy{simplify(polyObsDesired)}$$

Se $\phi(z) = \phi_d(z)$ logo

$$
\left\{
    \begin{array}{l}
        \sympy{polyObs.coeff(z,0)} = \sympy{polyObsDesired.coeff(z,0)}\\
        \sympy{polyObs.coeff(z,1)} = \sympy{polyObsDesired.coeff(z,1)}\\
        \sympy{polyObs.coeff(z,2)} = \sympy{polyObsDesired.coeff(z,2)}\\
    \end{array}
\right.
$$

Resolvendo o sistema temos $l_1 = \sympy{nl1} $ e $l_2 = \sympy{nl2}$

\subsection{Calculo ganho observador polos mais rápidos}

Supondo $L = \sympy{L}$

$$A_{ma} = \sympy{z*eye(2)} - \sympy{G} + \sympy{L}\sympy{C}$$
$$A_{ma} = \sympy{z*eye(2)} - \sympy{G -L*C}$$
$$A_{ma} = \sympy{z*eye(2) - (G -L*C)}$$

Logo temos que o polinômio característico para o sistema com realimentação é dado por 

$$\phi(z) = det\sympy{Amo}$$
$$\phi(z) = \sympy{collect(det(Amo),z)}$$

Temos que o polinômio caracteristico desejado é dado por

$$\phi_d(z) = \sympy{polyObsDesired2}$$
$$\phi_d(z) = \sympy{simplify(polyObsDesired2)}$$

Se $\phi(z) = \phi_d(z)$ logo

$$
\left\{
    \begin{array}{l}
        \sympy{polyObs.coeff(z,0)} = \sympy{polyObsDesired2.coeff(z,0)}\\
        \sympy{polyObs.coeff(z,1)} = \sympy{polyObsDesired2.coeff(z,1)}\\
        \sympy{polyObs.coeff(z,2)} = \sympy{polyObsDesired2.coeff(z,2)}\\
    \end{array}
\right.
$$

Resolvendo o sistema temos $l_1 = \sympy{nl21} $ e $l_2 = \sympy{nl22}$

\subsection{Comportamento em simulação}

Para simular o sistema sem o observador foi utilizado o simulink em conjunto com as representações mostradas nas figuras \ref{fig:e6-ss-model} e \ref{fig:e6-ssobs-model}, respectivamente para o sistema com e sem observador. Como forma de facilitar a identificação das partes funcionais foi adotado o seguinte esquema de cores: em verde os blocos referente a planta original, em amarelo o sinal de controle e em azul os blocos do observador.

\begin{figure}[H]
    \centering
    \includegraphics[width=0.9\linewidth]{img/exsim6_ss_model.png}
    \caption{Diagrama de Blocos do Sistema no Simulink do Controlador sem Observador}
    \label{fig:e6-ss-model}
\end{figure}

\begin{figure}[H]
    \centering
    \includegraphics[width=0.9\linewidth]{img/exsim6_ss_observer_model.png}
    \caption{Diagrama de Blocos do Sistema no Simulink do Controlador com Observador}
    \label{fig:e6-ssobs-model}
\end{figure}

\subsubsection{Sistema sem observador}

A partir da simulação, os dados foram exportados para uma variável e com isto foi obtido a seguinte resposta para sistema sem observador:

\begin{figure}[H]
    \centering
    \includegraphics[width=0.9\linewidth]{img/exsim6-ss-sim.png}
    \caption{Resposta Sistema controlador sem observador}
    \label{fig:e6-ss-sim}
\end{figure}

\begin{figure}[H]
    \centering
    \includegraphics[width=0.9\linewidth]{img/exsim6-ss-state-sim.png}
    \caption{Resposta Sistema controlador sem observador}
    \label{fig:e6-ss-state-sim}
\end{figure}

\subsubsection{Sistema com observador com polos iguais}

como efeito para o sistema sem observador foram obtidas as seguintes respostas

\begin{figure}[H]
    \centering
    \includegraphics[width=0.9\linewidth]{img/exsim6-ssobs10-sim.png}
    \caption{resposta sistema controlador com observador}
    \label{fig:e6-ssobs10-state-sim}
\end{figure}

definindo o erro do estimador de estados como $e[k] = x[k] - x_{obs}[k]$ temos

\begin{figure}[H]
    \centering
    \includegraphics[width=0.9\linewidth]{img/exsim6-ssobs10-err-sim.png}
    \caption{comportamento dinâmico do erro do observador}
\end{figure}

\subsubsection{Sistema com observador com polos mais rápidos}

como efeito para o sistema sem observador foram obtidas as seguintes respostas

\begin{figure}[H]
    \centering
    \includegraphics[width=0.9\linewidth]{img/exsim6-ssobs1-sim.png}
    \caption{resposta sistema controlador com observador}
\end{figure}

definindo o erro do estimador de estados como $e[k] = x[k] - x_{obs}[k]$ temos

\begin{figure}[H]
    \centering
    \includegraphics[width=0.9\linewidth]{img/exsim6-ssobs1-err-sim.png}
    \caption{comportamento dinâmico do erro do observador}
\end{figure}

\section{Conclusão}





% ------------------------------------------------------------------------------
\newpage
% Referências
\addcontentsline{toc}{section}{Referências} % Adiciona linha no indice
\bibliographystyle{abbrv} % Define Estilo e gera bibliografia
\bibliography{references} % Adiciona Arquivo com Referências

% Acrescentadas no arquivo references.bib
% para usa-las no texto basta usar \citep{}
% para citar sem usar no texto basta usar \nocite{}
\nocite{sympy}
\nocite{pythontex}
\nocite{matlabcontrol}
\nocite{matlabsymbolic}
\nocite{ogata2010modern}

% ------------------------------------------------------------------------------
\newpage
\section*{Anexos}
\addcontentsline{toc}{section}{Anexos} % Adiciona linha no indice
\subsection*{Python}

Para os cálculos e demonstrações foi utilizado o pacote \textit{Python}\TeX\ \cite{pythontex} para o \LaTeX\ em conjunto da bibliteca \textit{sympy}\cite{sympy}. Segue o script completo em python:

\inputminted[xleftmargin=15pt,linenos,frame=single,framesep=5pt,breaklines=true]{python}{../python/exsim6.py}

\newpage
\subsection*{Matlab}

\subsubsection*{Parte 1}
Para o desenho dos gráficos e simulações foi utilizado o \textit{Matlab} em conjunto das toolbox \textit{Control System}\cite{matlabcontrol} e \textit{Symbolic Math}\cite{matlabsymbolic}. Segue o código referente usado

\inputminted[xleftmargin=15pt,linenos,frame=single,framesep=5pt,breaklines=true]{matlab}{../matlab/exsim6/exsim6.m}


% ------------------------------------------------------------------------------
\end{document}
