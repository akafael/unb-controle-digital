% Pacotes e configurações padrão do estilo ``article''\
% -------------------------------------
\documentclass[a4paper,11pt]{article}
% Layout
% ------------------------------------------------------------------------------
\input{relat_layout.tex}

%\usepackage{circuitikz}
\usepackage[makestderr]{pythontex}

\title{Laboratório 3} % Define o título do Relatório
\author{Rafael Lima}

% Definições Auxiliares ( Macros próprias )
% ------------------------------------------------------------------------------
%\input{relat_aux.tex} % Arquivo com minhas macros
\newcommand{\npy}[1]{\sympy{round(#1,4)}}
% ----------------------------------~>ø<~---------------------------------------
\begin{document}
% Capa e Índice ----------------------------------------------------------------
\input{relat_capa.tex} % Capa para UnB
% Conteúdo ---------------------------------------------------------------------

\section{Controle Proporcional em Tempo Discreto}

% Código fonte colocado a parte para facilitar validação dentro do ipython
\begin{sympycode}
# Get Source Code
sys.path.insert(1, '../../')
from src.python.exsim3 import *
\end{sympycode}

Dado o sistema definido pela figura

\begin{figure}[H]
    \centering
    \includegraphics[width=0.8\linewidth]{img/exsim3modelPart1.png}
    \caption{Diagrama de bloco sistema 1}
\end{figure}

Definindo o controlador $G_c(s)$ por

\begin{equation}
    G_c(z) = \sympy{sGc}
\end{equation}

Temos que $G(z)$

\begin{equation}
    G(z) = \sympy{sGz}
\end{equation}

Logo a $G_{ma}(z)$

\begin{equation}\label{eq:ex3-gma1}
    G_{ma}(z) = \sympy{sGma}
\end{equation}

A partir do qual temos a seguinte função de transferência em malha fechada

\begin{equation}
    G_{mf}(z) = \sympy{sGmf}
\end{equation}

Em que o polinômio característico é dado por

\begin{equation}
    \phi(z) = \sympy{poly}
\end{equation}

Isolando $K$ temos

\begin{equation}\label{eq:ex3-k-poly}
    K(z) = \sympy{sK}
\end{equation}

\subsection{LGR}

Avaliando a função para diferentes períodos de amostragem obtemos os seguintes gráficos do lugar das raízes para a função de transferência de malha aberta $G_{ma}(z)$ dada pela equação \ref{eq:ex3-gma1}.

\begin{figure}[H]
    \centering
    \includegraphics[width=0.9\linewidth]{img/exsim3-rlocus-t500ms.png}
    \caption{ $T=0.5s$}
\end{figure}

\begin{figure}[H]
    \centering
    \includegraphics[width=0.9\linewidth]{img/exsim3-rlocus-t1000ms.png}
    \caption{ $T=1s$}
\end{figure}

\begin{figure}[H]
    \centering
    \includegraphics[width=0.9\linewidth]{img/exsim3-rlocus-t2000ms.png}
    \caption{ $T=2s$}
\end{figure}

Pelas figuras nota-se que o gráfico do LGR cruza o circulo de raio unitário definido em tracejado apenas no ponto $z=-1$ de modo que, em particular para este sistema podemos encontrar o valor do ganho crítico subsituindo $z=-1$ na equação \ref{eq:ex3-k-poly}. Em particular para $T=\{0.5,1,2\}$ temos os seguintes valores críticos para $K$:

\begin{table}[H]
    \centering
    $$
    \begin{array}{cl}
        \hline
        T [ms] & K_{max} \\
        \hline
        0.5 & \npy{sK.subs([(T,0.5),(z,-1)])}\\
        1 & \npy{sK.subs([(T,1),(z,-1)])}\\
        2 & \npy{sK.subs([(T,2),(z,-1)])}\\
        \hline
    \end{array}
    $$
\end{table}

\subsection{Resposta ao Degrau} 

Simulando o sistema para diferentes os períodos de amostragem $T={0.5,1,2}$ foram obtidos respectivavamente as seguintes respostas.

\begin{figure}[H]
    \centering
    \includegraphics[width=0.8\linewidth]{img/exsim3-step-t500ms.png}
    \caption{ Resposta ao Degrau ($T=0.5s$)}
\end{figure}

\begin{figure}[H]
    \centering
    \includegraphics[width=0.8\linewidth]{img/exsim3-step-t1000ms.png}
    \caption{ Resposta ao Degrau ($T=1s$)}
\end{figure}

\begin{figure}[H]
    \centering
    \includegraphics[width=0.8\linewidth]{img/exsim3-step-t2000ms.png}
    \caption{ Resposta ao Degrau ($T=2s$)}
\end{figure}

%% TODO Gerar tabela com Sobresinal e Tempo de Acomodação

\begin{table}[H]
    $$
    \begin{array}{ccccc}
        \hline
        T [s] & Polos & Sobressinal & \zeta & T_s\ [s] \\
        \hline
        0.5 & (0.4098+0.66227i, 0.4098-0.66227i) & 0.17 & 0.670 & 8.00\\
1.0 & (0.051819+0.60431i, 0.051819-0.60431i) & 0.21 & 0.694 & 8.00\\
2.0 & (-0.297+0.21709i, -0.297-0.21709i) & 0.73 & 0.807 & 8.00\\


        \\\hline
    \end{array}
    $$
\end{table}

\subsection{Resposta a Ramp Unitário}

Simulando o sistema para diferentes os períodos de amostragem $T={0.5,1,2}$ foram obtidos respectivavamente as seguintes respostas.

\begin{figure}[H]
    \centering
    \includegraphics[width=0.8\linewidth]{img/exsim3-ramp-t500ms.png}
    \caption{ Resposta ao Degrau ($T=0.5s$)}
\end{figure}

\begin{figure}[H]
    \centering
    \includegraphics[width=0.8\linewidth]{img/exsim3-ramp-t1000ms.png}
    \caption{ Resposta ao Degrau ($T=1s$)}
\end{figure}

\begin{figure}[H]
    \centering
    \includegraphics[width=0.8\linewidth]{img/exsim3-ramp-t2000ms.png}
    \caption{ Resposta ao Degrau ($T=2s$)}
\end{figure}

\section{Projeto usando LGR em Tempo Discreto}

Dado $G(s) = \frac{1}{s(s+a)}$. Discretizando no \textit{matlab} temos que


$$G(z) = G_{zoh}(z)\mathcal{Z}\left\{G(s)\right\}$$
$$G(z) = (1-z^{-1})\mathcal{Z}\left\{\frac{1}{4\,\left(s+2\right)}-\frac{1}{4\,s}+\frac{1}{2\,s^2}\right\} $$
$$G(z) = (1-z^{-1})\mathcal{Z}\left\{\frac{1}{4\,\left(s+2\right)}-\frac{1}{4\,s}+\frac{1}{2\,s^2}\right\} $$
$$G(z) = \sympy{sGz2}$$
$$G(z) = \sympy{simplifyFraction(sGz2,z)}$$

Em particular para $T=0.2$ avaliando a expressão no matlab temos:

$$G(z),\frac{\frac{2533546664982251\,z}{144115188075855872}+\frac{554410548014771}{36028797018963968}}{z^2-\frac{3761226368457787\,z}{2251799813685248}+\frac{6037706219090157}{9007199254740992}}
$$

\subsection{Avaliação dos Requisitos}

Para este controlador temos como requisito $E=0.5$ e tempo de acomodação $t_s = 0.2s$. Dado um sistema de segunda ordem com a função de transferência em malha aberta na forma:

\begin{equation}
    G(s) = \frac{\omega_n^2}{s^2 + 2\zeta\omega_n + \omega_n^2}
\end{equation}

Temos as seguintes relações no plano $s$: tempo de acomodação pode ser obtido por $t_s = \frac{4}{\zeta\omega}$
o os polos por $ \rho = -\zeta\omega_n \pm j\omega_n\sqrt{1-\zeta^2}$. Sabendo que a relação enre plano $z$ e plano $s$ é $z=e^{-sT}$, temos que os polos desejados no plano $z$ serâo:

\begin{equation}
    z_{\rho} = e^{-sT}|_{s=\rho} =  e^{-\zeta\omega_n T} e^{\pm j\omega_n\sqrt{1-\zeta^2}}
\end{equation}

Desta forma temos que o módulo será $|z_{\rho}| =  e^{-\zeta\omega_n T} = e^{-\frac{4T}{t_s}}$ e o fase de $z_\rho$ será $arg(z_{\rho}) = \omega_n\sqrt{1-\zeta^2} = \frac{4}{t_s\zeta}\sqrt{1-\zeta^2}$. Com isto, para garantir que este requisitos seja cumpridos basta verificar se o LGR no plano $z$ passa pelo ponto $z = z_{\rho}$.

Resolvedo a expressão teremos os seguintes polos desejados no plano $s$

$$
\left\{
\begin{array}{l}
    s_0 = \sympy{round(desiredPoles[0],4)}\\
    s_1 = \sympy{round(desiredPoles[1],4)}
\end{array}
\right.
$$

Transformando para o polo $z$ temos

$$
\left\{
\begin{array}{l}
    z_0 = e^{s_0 T} = \sympy{round(desiredPolesZ[0],4)}\\
    z_1 = e^{s_1 T} = \sympy{round(desiredPolesZ[1],4)}
\end{array}
\right.
$$

\subsection{Resposta sistema em malha fechada}

% Calcular função de transferẽncia

Inicialmente, para uma primeira estimativa do comportamento do sistema podemos obter o gráfico do lugar das raizes para o sistema em malha fechada sem o controlador. O que seria equivalente a um controlador dado apenas por um ganho ajustável.

\begin{figure}[H]
    \centering
    \includegraphics[width=0.9\linewidth]{img/exsim3-rlocus-g2.png}
    \caption{LGR sem o controlador}
    \label{fig:ex3-rlocus-g2}
\end{figure}

A partir do gráfico \ref{fig:ex3-rlocus-g2} temos que o LGR da planta sem o controlador já intercepta a curva dada por $\zeta = 0.5$, no entanto não passa pelos polos desejados marcados em $*$. Uma forma simples de deslocar o gráfico do LGR lateralmente. Para tal um controlador possível é

\begin{equation}
    Gc(z) = K\frac{(z-e^{-aT})}{z-\beta} = \sympy{sGc2}
\end{equation}

\begin{equation}
    Gc(z)Gz(z) = \sympy{sGma2}
\end{equation}

\begin{equation}
    G_{mf}(z) = \sympy{sGmf2}
\end{equation}

Em que o polinômio característico é dado por

\begin{equation}
    \phi(z) = \sympy{collect(poly2,K)}
\end{equation}

Como os polos desejado são raizes da polinômio característico, podemos obter $K$ e $\sympy{b}$ a partir do seguinte sistema:

$$
\left\{
\begin{array}{l}
    \phi(z_0) = \sympy{roundExpr(sysKb[0],4)} = 0\\
    \phi(z_1) = \sympy{roundExpr(sysKb[1],4)} = 0
\end{array}
\right.
$$

A partir do controlador projetado temos a seguinte representação no \textit{Simulink} em diagrama de blocos para o sistema em malha fechada.

\begin{figure}[H]
    \centering
    \includegraphics[width=0.8\linewidth]{img/exsim3modelPart2.png}
    \caption{Diagrama de bloco sistema em Malha fechada com controlador}
\end{figure}

Resolvendo o sistema temos $K = \npy{nK}$ e $\sympy{b} = \npy{nb}$. De modo que função de malha fechada fica

\begin{equation}
    G_{mf}(z) = \sympy{roundExpr(nGmf2,4)}
\end{equation}

Com o controlador proposto passamos a ter o seguinte LGR:

\begin{figure}[H]
    \centering
    \includegraphics[width=0.8\linewidth]{img/exsim3-rlocus-g2-control.png}
    \caption{LGR com o controlador}
    \label{fig:ex3-rlocus-g2-control}
\end{figure}

A resposta ao degrau unitário passa a ser a seguinte:

\begin{figure}[H]
    \centering
    \includegraphics[width=0.8\linewidth]{img/exsim3-step-g2-control.png}
    \caption{Resposta ao Degrau}
    \label{fig:ex3-step-g2-control}
\end{figure}

A resposta a rampa unitária passa a ser a seguinte:

\begin{figure}[H]
    \centering
    \includegraphics[width=0.8\linewidth]{img/exsim3-ramp-g2-control.png}
    \caption{Resposta a Rampa unitária}
    \label{fig:ex3-step-g2-control}
\end{figure}

\subsection{Projeto Controlador}

Embora o controlador proposto definido garanta os requisitos da resposta transiente do sistema, podemos notar pela figura \ref{fig:ex3-step-g2-control} que erro em regime permanente para a resposta ao degrau não é nulo. Desta forma é necessário uma modificação para garantir as que tanto os requisitos em regime permanente como em regime estácionário sejam cumpridos. Uma possível solução é acrescentar um atraso com ganho unitário e polos não dominantes no sistema, isto é próximos de $z=0$ desta forma o erro estacionário pode ser corrigido sem alterar muito a dinâmica do sistema, permitindo reaproveitar o controlador desenvolvido. Para tal é proposto um controlador na seguinte forma:

\begin{equation}
    G_{c2}(z) = \frac{z - \gamma K_c}{z-\gamma}\ ,\ \gamma << 1
\end{equation}

\begin{figure}[H]
    \centering
    \includegraphics[width=0.8\linewidth]{img/exsim3modelPart2d.png}
    \caption{Diagrama de bloco sistema em Malha fechada com controlador com correção de erro estacionário}
\end{figure}

Para garantir uma redução em um terço podemos escolher $K_c = 1/(1/3) = 3$ e $\gamma = 0.01$. A resposta ao degrau unitário passa a ser a seguinte:

\begin{figure}[H]
    \centering
    \includegraphics[width=0.8\linewidth]{img/exsim3-step-g3-control.png}
    \caption{Resposta ao Degrau}
    \label{fig:ex3-step-g2-control}
\end{figure}

A resposta a rampa unitária passa a ser a seguinte:

\begin{figure}[H]
    \centering
    \includegraphics[width=0.8\linewidth]{img/exsim3-ramp-g3-control.png}
    \caption{Resposta a Rampa unitária}
    \label{fig:ex3-step-g2-control}
\end{figure}

No que o erro para rampa diminuiu mas a resposta ao transiente foi comprometida.

\section{Conclusão}

Foi possível simular todos os controladores e com auxílio da biblioteca simbólica obter a resposta exata para os polos desejado para o sistema em malha fechada. A simulação a partir do simulink e a resposta a partir das funções $lsim$ e $step$ foram diferentes. Enquanto para o simulink a saída foi representada como um sinal constante, com as funções o sinal de saída também era amostrado.

% ------------------------------------------------------------------------------
\newpage
% Referências
\addcontentsline{toc}{section}{Referências} % Adiciona linha no indice
\bibliographystyle{abbrv} % Define Estilo e gera bibliografia
\bibliography{references} % Adiciona Arquivo com Referências

% Acrescentadas no arquivo references.bib
% para usa-las no texto basta usar \citep{}
% para citar sem usar no texto basta usar \nocite{}
\nocite{sympy}
\nocite{pythontex}
\nocite{matlabcontrol}
\nocite{matlabsymbolic}

% ------------------------------------------------------------------------------
\newpage
\section*{Anexos}
\addcontentsline{toc}{section}{Anexos} % Adiciona linha no indice
\subsection*{Python}

Para os cálculos e demonstrações foi utilizado o pacote \textit{Python}\TeX\ \cite{pythontex} para o \LaTeX\ em conjunto da bibliteca \textit{sympy}\cite{sympy}. Segue o script completo em python:

\inputminted[xleftmargin=15pt,linenos,frame=single,framesep=5pt,breaklines=true]{python}{../python/exsim3.py}

\newpage
\subsection*{Matlab}

%\subsubsection*{Parte 1}
Para o desenho dos gráficos e simulações foi utilizado o \textit{Matlab} em conjunto das toolbox \textit{Control System}\cite{matlabcontrol} e \textit{Symbolic Math}\cite{matlabsymbolic}. Segue o código referente usado

\inputminted[xleftmargin=15pt,linenos,frame=single,framesep=5pt,breaklines=true]{matlab}{../matlab/exsim3.m}

%\subsubsection*{Parte 2}
%Na segunda parte foi utilizado uma versão modificada do script em \textit{Matlab} fornecido pelo professor:
%\inputminted[xleftmargin=15pt,linenos,frame=single,framesep=5pt]{matlab}{../matlab/exsim2/exsim2script.m}



% ------------------------------------------------------------------------------
\end{document}
