% Pacotes e configurações padrão do estilo ``article''\
% -------------------------------------
\documentclass[a4paper,11pt]{article}
% Layout
% ------------------------------------------------------------------------------
\input{relat_layout.tex}

%\usepackage{circuitikz}
\usepackage[makestderr]{pythontex}

\title{Laboratório 3} % Define o título do Relatório
\author{Rafael Lima}

% Definições Auxiliares ( Macros próprias )
% ------------------------------------------------------------------------------
%\input{relat_aux.tex} % Arquivo com minhas macros
\newcommand{\npy}[1]{\sympy{round(#1,4)}}
% ----------------------------------~>ø<~---------------------------------------
\begin{document}
% Capa e Índice ----------------------------------------------------------------
\input{relat_capa.tex} % Capa para UnB
% Conteúdo ---------------------------------------------------------------------

\section{Controle Proporcional em Tempo Discreto}

% Código fonte colocado a parte para facilitar validação dentro do ipython
\begin{sympycode}
# Get Source Code
sys.path.insert(1, '../../')
from src.python.exsim3 import *
\end{sympycode}

Dado o controlador $G_c(s)$ definido por

\begin{equation}
    G_c(z) = \sympy{sGc}
\end{equation}

Temos que $G(z)$

\begin{equation}
    G(z) = \sympy{sGz}
\end{equation}

Logo a $G_{ma}(z)$

\begin{equation}\label{eq:ex3-gma1}
    G_{ma}(z) = \sympy{sGma}
\end{equation}

A partir do qual temos a seguinte função de transferência em malha fechada

\begin{equation}
    G_{mf}(z) = \sympy{sGmf}
\end{equation}

Em que o polinômio característico é dado por

\begin{equation}
    \phi(z) = \sympy{poly}
\end{equation}

Isolando $K$ temos

\begin{equation}\label{eq:ex3-k-poly}
    K(z) = \sympy{sK}
\end{equation}

\subsection{LGR}

Avaliando a função para diferentes períodos de amostragem obtemos os seguintes gráficos do lugar das raízes para a função de transferência de malha aberta $G_{ma}(z)$ dada pela equação \ref{eq:ex3-gma1}.

\begin{figure}[H]
    \centering
    \includegraphics[width=0.9\linewidth]{img/exsim3-rlocus-t500ms.png}
    \caption{ $T=0.5s$}
\end{figure}

\begin{figure}[H]
    \centering
    \includegraphics[width=0.9\linewidth]{img/exsim3-rlocus-t1000ms.png}
    \caption{ $T=1s$}
\end{figure}

\begin{figure}[H]
    \centering
    \includegraphics[width=0.9\linewidth]{img/exsim3-rlocus-t2000ms.png}
    \caption{ $T=2s$}
\end{figure}

Pelas figuras nota-se que o gráfico do LGR cruza o circulo de raio unitário definido em tracejado apenas no ponto $z=-1$ de modo que, em particular para este sistema podemos encontrar o valor do ganho crítico subsituindo $z=-1$ na equação \ref{eq:ex3-k-poly}. Em particular para $T=\{0.5,1,2\}$ temos os seguintes valores críticos para $K$:

\begin{table}[H]
    \centering
    $$
    \begin{array}{cl}
        \hline
        T [ms] & K_{max} \\
        \hline
        0.5 & \npy{sK.subs([(T,0.5),(z,-1)])}\\
        1 & \npy{sK.subs([(T,1),(z,-1)])}\\
        2 & \npy{sK.subs([(T,2),(z,-1)])}\\
        \hline
    \end{array}
    $$
\end{table}

\subsection{Resposta ao Degrau} 

Simulando o sistema para diferentes os períodos de amostragem $T={0.5,1,2}$ foram obtidos respectivavamente as seguintes respostas.

\begin{figure}[H]
    \centering
    \includegraphics[width=0.8\linewidth]{img/exsim3-step-t500ms.png}
    \caption{ Resposta ao Degrau ($T=0.5s$)}
\end{figure}

\begin{figure}[H]
    \centering
    \includegraphics[width=0.8\linewidth]{img/exsim3-step-t1000ms.png}
    \caption{ Resposta ao Degrau ($T=1s$)}
\end{figure}

\begin{figure}[H]
    \centering
    \includegraphics[width=0.8\linewidth]{img/exsim3-step-t2000ms.png}
    \caption{ Resposta ao Degrau ($T=2s$)}
\end{figure}

%% TODO Gerar tabela com Sobresinal e Tempo de Acomodação

\begin{table}[H]
    $$
    \begin{array}{cccc}
        \hline
        T [s] & Sobressinal & \zeta & T_s\ [s] \\
        \hline
        0.5 & (0.4098+0.66227i, 0.4098-0.66227i) & 0.17 & 0.670 & 8.00\\
1.0 & (0.051819+0.60431i, 0.051819-0.60431i) & 0.21 & 0.694 & 8.00\\
2.0 & (-0.297+0.21709i, -0.297-0.21709i) & 0.73 & 0.807 & 8.00\\


        \\\hline
    \end{array}
    $$
\end{table}


\subsection{}

\section{Projeto usando LGR em Tempo Discreto}

\subsection{Avaliação dos Requisitos}

Para este controlador temos como requisito $E=0.5$ e tempo de acomodação $t_s = 0.2s$ dado um sistema de segunda ordem com a função de transferência em malha aberta na forma:

\begin{equation}
    G(s) = \frac{\omega_n^2}{s^2 + 2\zeta\omega_n + \omega_n^2}
\end{equation}

Temos as seguintes relações no plano $s$, temos que tempo de acomodação é dado por $t_s = \frac{4}{\zeta\omega}$
o os polos podem ser obtidos por $ \rho = -\zeta\omega_n \pm j\omega_n\sqrt{1-\zeta^2}$. Sabendo que a relação enre plano $z$ e plano $s$ é $z=e^{-sT}$, temos que os polos desejados no plano $z$ serâo:

\begin{equation}
    z_{\rho} = e^{-sT}|_{s=\rho} =  e^{-\zeta\omega_n T} e^{\pm j\omega_n\sqrt{1-\zeta^2}}
\end{equation}

Desta forma temos que o módulo será $|z_{\rho}| =  e^{-\zeta\omega_n T} = e^{-\frac{4T}{t_s}}$ e o fase de $z_\rho$ será $arg(z_{\rho}) = \omega_n\sqrt{1-\zeta^2} = \frac{4}{t_s\zeta}\sqrt{1-\zeta^2}$. Com isto, para garantir que este requisitos seja cumpridos basta verificar se o LGR no plano $z$ passa pelo ponto $z = z_{\rho}$.

\subsection{Resposta sistema em malha fechada}

% Calcular função de transferẽncia

Inicialmente, para uma primeira estimativa do comportamento do sistema podemos obter o gráfico do lugar das raizes para o sistema em malha fechada sem o controlador. O que seria equivalente a um controlador dado apenas por um ganho ajustável.

\begin{figure}[H]
    \centering
    \includegraphics[width=0.9\linewidth]{img/exsim3-rlocus-g2.png}
    \caption{LGR sem o controlador}
    \label{fig:ex3-rlocus-g2}
\end{figure}

A partir do gráfico \ref{fig:ex3-rlocus-g2} temos que o LGR da planta sem o controlador já intercepta a curva dada por $\zeta = 0.5$, no entanto é necessário verificar se o tempo de acomodação já cumpre os requisitos.

%%\section{Conclusão}

% ------------------------------------------------------------------------------

% Referências
\addcontentsline{toc}{section}{Referências} % Adiciona linha no indice
\bibliographystyle{abbrv} % Define Estilo e gera bibliografia
\bibliography{references} % Adiciona Arquivo com Referências

% Acrescentadas no arquivo references.bib
% para usa-las no texto basta usar \citep{}
% para citar sem usar no texto basta usar \nocite{}
\nocite{sympy}
\nocite{pythontex}
\nocite{matlabcontrol}
\nocite{matlabsymbolic}

% ------------------------------------------------------------------------------
\newpage
\section*{Anexos}
\addcontentsline{toc}{section}{Anexos} % Adiciona linha no indice
\subsection*{Python}

Para os cálculos e demonstrações foi utilizado o pacote \textit{Python}\TeX\ \cite{pythontex} para o \LaTeX\ em conjunto da bibliteca \textit{sympy}\cite{sympy}. Segue o script completo em python:

\inputminted[xleftmargin=15pt,linenos,frame=single,framesep=5pt,breaklines=true]{python}{../python/exsim3.py}

\newpage
\subsection*{Matlab}

%\subsubsection*{Parte 1}
Para o desenho dos gráficos e simulações foi utilizado o \textit{Matlab} em conjunto das toolbox \textit{Control System}\cite{matlabcontrol} e \textit{Symbolic Math}\cite{matlabsymbolic}. Segue o código referente usado

\inputminted[xleftmargin=15pt,linenos,frame=single,framesep=5pt,breaklines=true]{matlab}{../matlab/exsim3.m}

%\subsubsection*{Parte 2}
%Na segunda parte foi utilizado uma versão modificada do script em \textit{Matlab} fornecido pelo professor:
%\inputminted[xleftmargin=15pt,linenos,frame=single,framesep=5pt]{matlab}{../matlab/exsim2/exsim2script.m}



% ------------------------------------------------------------------------------
\end{document}
